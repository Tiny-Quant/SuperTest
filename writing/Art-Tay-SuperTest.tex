% Options for packages loaded elsewhere
\PassOptionsToPackage{unicode}{hyperref}
\PassOptionsToPackage{hyphens}{url}
\PassOptionsToPackage{dvipsnames,svgnames,x11names}{xcolor}
%
\documentclass[
  11pt,
  letterpaper,
]{article}

\usepackage{amsmath,amssymb}
\usepackage{iftex}
\ifPDFTeX
  \usepackage[T1]{fontenc}
  \usepackage[utf8]{inputenc}
  \usepackage{textcomp} % provide euro and other symbols
\else % if luatex or xetex
  \usepackage{unicode-math}
  \defaultfontfeatures{Scale=MatchLowercase}
  \defaultfontfeatures[\rmfamily]{Ligatures=TeX,Scale=1}
\fi
\usepackage{lmodern}
\ifPDFTeX\else  
    % xetex/luatex font selection
\fi
% Use upquote if available, for straight quotes in verbatim environments
\IfFileExists{upquote.sty}{\usepackage{upquote}}{}
\IfFileExists{microtype.sty}{% use microtype if available
  \usepackage[]{microtype}
  \UseMicrotypeSet[protrusion]{basicmath} % disable protrusion for tt fonts
}{}
\makeatletter
\@ifundefined{KOMAClassName}{% if non-KOMA class
  \IfFileExists{parskip.sty}{%
    \usepackage{parskip}
  }{% else
    \setlength{\parindent}{0pt}
    \setlength{\parskip}{6pt plus 2pt minus 1pt}}
}{% if KOMA class
  \KOMAoptions{parskip=half}}
\makeatother
\usepackage{xcolor}
\usepackage[margin=1in]{geometry}
\setlength{\emergencystretch}{3em} % prevent overfull lines
\setcounter{secnumdepth}{-\maxdimen} % remove section numbering
% Make \paragraph and \subparagraph free-standing
\ifx\paragraph\undefined\else
  \let\oldparagraph\paragraph
  \renewcommand{\paragraph}[1]{\oldparagraph{#1}\mbox{}}
\fi
\ifx\subparagraph\undefined\else
  \let\oldsubparagraph\subparagraph
  \renewcommand{\subparagraph}[1]{\oldsubparagraph{#1}\mbox{}}
\fi


\providecommand{\tightlist}{%
  \setlength{\itemsep}{0pt}\setlength{\parskip}{0pt}}\usepackage{longtable,booktabs,array}
\usepackage{calc} % for calculating minipage widths
% Correct order of tables after \paragraph or \subparagraph
\usepackage{etoolbox}
\makeatletter
\patchcmd\longtable{\par}{\if@noskipsec\mbox{}\fi\par}{}{}
\makeatother
% Allow footnotes in longtable head/foot
\IfFileExists{footnotehyper.sty}{\usepackage{footnotehyper}}{\usepackage{footnote}}
\makesavenoteenv{longtable}
\usepackage{graphicx}
\makeatletter
\def\maxwidth{\ifdim\Gin@nat@width>\linewidth\linewidth\else\Gin@nat@width\fi}
\def\maxheight{\ifdim\Gin@nat@height>\textheight\textheight\else\Gin@nat@height\fi}
\makeatother
% Scale images if necessary, so that they will not overflow the page
% margins by default, and it is still possible to overwrite the defaults
% using explicit options in \includegraphics[width, height, ...]{}
\setkeys{Gin}{width=\maxwidth,height=\maxheight,keepaspectratio}
% Set default figure placement to htbp
\makeatletter
\def\fps@figure{htbp}
\makeatother
\newlength{\cslhangindent}
\setlength{\cslhangindent}{1.5em}
\newlength{\csllabelwidth}
\setlength{\csllabelwidth}{3em}
\newlength{\cslentryspacingunit} % times entry-spacing
\setlength{\cslentryspacingunit}{\parskip}
\newenvironment{CSLReferences}[2] % #1 hanging-ident, #2 entry spacing
 {% don't indent paragraphs
  \setlength{\parindent}{0pt}
  % turn on hanging indent if param 1 is 1
  \ifodd #1
  \let\oldpar\par
  \def\par{\hangindent=\cslhangindent\oldpar}
  \fi
  % set entry spacing
  \setlength{\parskip}{#2\cslentryspacingunit}
 }%
 {}
\usepackage{calc}
\newcommand{\CSLBlock}[1]{#1\hfill\break}
\newcommand{\CSLLeftMargin}[1]{\parbox[t]{\csllabelwidth}{#1}}
\newcommand{\CSLRightInline}[1]{\parbox[t]{\linewidth - \csllabelwidth}{#1}\break}
\newcommand{\CSLIndent}[1]{\hspace{\cslhangindent}#1}

\usepackage{booktabs}
\usepackage{longtable}
\usepackage{array}
\usepackage{multirow}
\usepackage{wrapfig}
\usepackage{float}
\usepackage{colortbl}
\usepackage{pdflscape}
\usepackage{tabu}
\usepackage{threeparttable}
\usepackage{threeparttablex}
\usepackage[normalem]{ulem}
\usepackage{makecell}
\usepackage{xcolor}
\makeatletter
\makeatother
\makeatletter
\makeatother
\makeatletter
\@ifpackageloaded{caption}{}{\usepackage{caption}}
\AtBeginDocument{%
\ifdefined\contentsname
  \renewcommand*\contentsname{Table of contents}
\else
  \newcommand\contentsname{Table of contents}
\fi
\ifdefined\listfigurename
  \renewcommand*\listfigurename{List of Figures}
\else
  \newcommand\listfigurename{List of Figures}
\fi
\ifdefined\listtablename
  \renewcommand*\listtablename{List of Tables}
\else
  \newcommand\listtablename{List of Tables}
\fi
\ifdefined\figurename
  \renewcommand*\figurename{Figure}
\else
  \newcommand\figurename{Figure}
\fi
\ifdefined\tablename
  \renewcommand*\tablename{Table}
\else
  \newcommand\tablename{Table}
\fi
}
\@ifpackageloaded{float}{}{\usepackage{float}}
\floatstyle{ruled}
\@ifundefined{c@chapter}{\newfloat{codelisting}{h}{lop}}{\newfloat{codelisting}{h}{lop}[chapter]}
\floatname{codelisting}{Listing}
\newcommand*\listoflistings{\listof{codelisting}{List of Listings}}
\makeatother
\makeatletter
\@ifpackageloaded{caption}{}{\usepackage{caption}}
\@ifpackageloaded{subcaption}{}{\usepackage{subcaption}}
\makeatother
\makeatletter
\@ifpackageloaded{tcolorbox}{}{\usepackage[skins,breakable]{tcolorbox}}
\makeatother
\makeatletter
\@ifundefined{shadecolor}{\definecolor{shadecolor}{rgb}{.97, .97, .97}}
\makeatother
\makeatletter
\makeatother
\makeatletter
\makeatother
\ifLuaTeX
  \usepackage{selnolig}  % disable illegal ligatures
\fi
\IfFileExists{bookmark.sty}{\usepackage{bookmark}}{\usepackage{hyperref}}
\IfFileExists{xurl.sty}{\usepackage{xurl}}{} % add URL line breaks if available
\urlstyle{same} % disable monospaced font for URLs
\hypersetup{
  pdftitle={Qualifying Exam},
  pdfauthor={Art Tay},
  colorlinks=true,
  linkcolor={blue},
  filecolor={Maroon},
  citecolor={Blue},
  urlcolor={Blue},
  pdfcreator={LaTeX via pandoc}}

\title{Qualifying Exam}
\author{Art Tay}
\date{}

\begin{document}
\maketitle
\ifdefined\Shaded\renewenvironment{Shaded}{\begin{tcolorbox}[boxrule=0pt, sharp corners, interior hidden, borderline west={3pt}{0pt}{shadecolor}, enhanced, frame hidden, breakable]}{\end{tcolorbox}}\fi

\hypertarget{introduction}{%
\section{Introduction}\label{introduction}}

\begin{itemize}
\item
  Overview of the problem area.

  \begin{itemize}
  \tightlist
  \item
    Graphs are an important data structure.

    \begin{itemize}
    \tightlist
    \item
      GRAPHS provide an incredibly flexible structure for modeling
      complex data. Data can naturally appear as graphs, like molecules.
      We can reduce data to a graph, such as the key points of a image.
      We can even use graphs to add structure, such as grammatical
      relationships.
    \end{itemize}
  \item
    GNN models are good at prediction and inference on graph data.

    \begin{itemize}
    \tightlist
    \item
      Graph Neural Networks (GNNs) have become a popular choice for
      prediction and inference on graph data. At their core, GNNs work
      by iteratively updating node embeddings based on information from
      neighboring nodes. The idea is to use the graph's structure to
      engineer better features. This message passing scheme allows GNNs
      to capture complex dependencies and patterns present within the
      graph structure. GNN architectures typically consist of multiple
      layers, each performing message passing and aggregation operations
      to refine the embeddings. These layers are often followed by
      pooling and dense prediction layers to produce the final output.
    \end{itemize}
  \item
    There are many important applications for graph classification
    models.

    \begin{itemize}
    \tightlist
    \item
      Some important applications of graph classification include
      predicting chemical toxicity (Bai et al. 2019), classifying
      proteins (Gallicchio and Micheli 2019), and even detecting cancer
      from pathology slides (Xiao et al. 2023).
    \end{itemize}
  \item
    \textbf{Problem:} While GNNs achieve remarkable predictive power,
    their complexity prevents the exaction of the scientific rationale.
  \end{itemize}
\item
  Why is the problem important?

  \begin{itemize}
  \tightlist
  \item
    Explaining or interpreting GNN predictions would

    \begin{itemize}
    \tightlist
    \item
      help with the adoption of such models for critical applications,
    \item
      prevent adversarial attacks,
    \item
      detect potential implicit discrimination,
    \item
      guide scientific as well as machine learning research.
    \end{itemize}
  \end{itemize}
\item
  How does the problem relate to the fundamentals areas of Statistics?

  \begin{itemize}
  \item
    Explain-ability vs Interpretability

    \begin{itemize}
    \tightlist
    \item
      Yuan et al. (2022)
    \item
      A model is interpretable if the models decision process can be
      readily understood by humans. For example, a linear regression
      model is interpretable because the coefficient clearly define how
      any prediction get made.
    \item
      A model is explainable if the models prediction can be reasoned
      post-hoc. Permuting each variable and measuring the variation in
      the predictions can be used to estimate each variables marginal
      effect {[}cite{]}.
    \end{itemize}
  \item
    One goal would be to create a GNN type model whose decision process
    is human interpretable. A straight translation from statistics would
    be a circuit type analysis {[}cite{]}. For graphs, this would mean
    some form of coefficients on subgraphs producing the prediction.
  \item
    Another goal might be to develope a method that determines if a
    feature is statistical significant to the GNN model. The challenge
    is that the graph features that matter to researchers aren't
    necessarily tabular.
  \end{itemize}
\item
  What is the impact of solving this problem?

  \begin{itemize}
  \item
    In the application where GNNs have shown strong predictive power, we
    can exact a testable scientific hypothesis for the nature of the
    classification.
  \item
    In the application where GNNs have weak predictive power, highlight
    the potential misunderstandings the model is having.
  \end{itemize}
\end{itemize}

\hypertarget{notation}{%
\section{Notation}\label{notation}}

\hypertarget{analysis-of-core-papers}{%
\section{Analysis of Core Papers}\label{analysis-of-core-papers}}

\hypertarget{gnninterpreter}{%
\subsection{GNNInterpreter}\label{gnninterpreter}}

(Wang and Shen 2024)

\begin{itemize}
\item
  Note on model-level explanations.
\item
  Prediction objective.
\item
  Embedding objective.
\item
  Intuitive explanation of concrete distribution.
\item
  Regularization terms.
\end{itemize}

\hypertarget{d4explainer}{%
\subsection{D4Explainer}\label{d4explainer}}

(Chen et al. 2023)

\begin{itemize}
\item
  Note on counter-factual explanations.
\item
  Graph diffusion.
\end{itemize}

\hypertarget{protgnn}{%
\subsection{ProtGNN}\label{protgnn}}

(Zhang et al. 2021)

\hypertarget{synthesis-of-core-papers}{%
\section{Synthesis of Core Papers}\label{synthesis-of-core-papers}}

\begin{itemize}
\tightlist
\item
  Comparison of generation methods.

  \begin{itemize}
  \tightlist
  \item
    GNNInterpreter uses continuously relaxed discrete distributions.
  \item
    D4Explainer uses diffusion.
  \item
    Diffusion is slower, but can be more realistic. Probably because
    diffusion is less subject to the \textbf{out-of-distribution (OOD)
    problem}.
  \item
    Prototype projection are like generative methods. Restricted to in
    distribution, but realism is all but guaranteed.
  \end{itemize}
\end{itemize}

\hypertarget{technical-details}{%
\section{Technical Details}\label{technical-details}}

\begin{itemize}
\tightlist
\item
  Minimal reproduction of each method on MUTAG.
\end{itemize}

\hypertarget{future-directions}{%
\section{Future Directions}\label{future-directions}}

\hypertarget{references}{%
\section*{References}\label{references}}
\addcontentsline{toc}{section}{References}

\hypertarget{refs}{}
\begin{CSLReferences}{1}{0}
\leavevmode\vadjust pre{\hypertarget{ref-bai2019unsupervised}{}}%
Bai, Yunsheng, Hao Ding, Yang Qiao, Agustin Marinovic, Ken Gu, Ting
Chen, Yizhou Sun, and Wei Wang. 2019. {``Unsupervised Inductive
Graph-Level Representation Learning via Graph-Graph Proximity.''}
\url{https://arxiv.org/abs/1904.01098}.

\leavevmode\vadjust pre{\hypertarget{ref-Chen_Wu_Gupta_Ying_2023}{}}%
Chen, Jialin, Shirley Wu, Abhijit Gupta, and Rex Ying. 2023.
{``D4Explainer: In-Distribution GNN Explanations via Discrete Denoising
Diffusion,''} no. arXiv:2310.19321 (October).
\url{https://doi.org/10.48550/arXiv.2310.19321}.

\leavevmode\vadjust pre{\hypertarget{ref-gallicchio2019fast}{}}%
Gallicchio, Claudio, and Alessio Micheli. 2019. {``Fast and Deep Graph
Neural Networks.''} \url{https://arxiv.org/abs/1911.08941}.

\leavevmode\vadjust pre{\hypertarget{ref-Wang_Shen_2024}{}}%
Wang, Xiaoqi, and Han-Wei Shen. 2024. {``GNNInterpreter: A Probabilistic
Generative Model-Level Explanation for Graph Neural Networks,''} no.
arXiv:2209.07924 (February).
\url{https://doi.org/10.48550/arXiv.2209.07924}.

\leavevmode\vadjust pre{\hypertarget{ref-Xiao_Wang_Rong_Yang_Zhang_Zhan_Bishop_Wilhelm_Zhang_Pickering_et_al._2023}{}}%
Xiao, Guanghua, Shidan Wang, Ruichen Rong, Donghan Yang, Xinyi Zhang,
Xiaowei Zhan, Justin Bishop, et al. 2023. \emph{Deep Learning of Cell
Spatial Organizations Identifies Clinically Relevant Insights in Tissue
Images}. Preprint. In Review.
\url{https://doi.org/10.21203/rs.3.rs-2928838/v1}.

\leavevmode\vadjust pre{\hypertarget{ref-Yuan_Yu_Gui_Ji_2022}{}}%
Yuan, Hao, Haiyang Yu, Shurui Gui, and Shuiwang Ji. 2022.
{``Explainability in Graph Neural Networks: A Taxonomic Survey,''} no.
arXiv:2012.15445 (July).
\url{https://doi.org/10.48550/arXiv.2012.15445}.

\leavevmode\vadjust pre{\hypertarget{ref-Zhang_Liu_Wang_Lu_Lee_2021}{}}%
Zhang, Zaixi, Qi Liu, Hao Wang, Chengqiang Lu, and Cheekong Lee. 2021.
{``ProtGNN: Towards Self-Explaining Graph Neural Networks,''} no.
arXiv:2112.00911 (December).
\url{https://doi.org/10.48550/arXiv.2112.00911}.

\end{CSLReferences}



\end{document}
